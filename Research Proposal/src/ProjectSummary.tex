\section{Project Summary}

Infectious diseases, and particuarly epidemics, are an area of high interest both in terms of what makes the disease and how it spreads. While vaccines are useful to prevent infection they are often of little use once the infection has taken hold. The development of antiviral drugs has been shown to reduce the period of time that an individual is infectious \cite{Stiver:2003}, and also reduce the infectiveness and transmissibility of each individual \cite{Hayden:2004}. It is clear then, that intervention by antivirals would be desired in the event of an influenza epidemic.

It is sensible to think that the earlier a person who is infected begins taking antivirals, the better the effect will be. This has been shown by \cite{Black:2013}, where it was demonstrated that the `doubling time' of an epidemic (that is, the expected amount of time for the number of infected people to double) is quite sensitive to the delay before an infected person receives antivirals.

This project aims to investigate what effect pre-distributing antivirals to households before they become infected would have during an epidemic, and also to attempt to derive conditions for which pre-distributing is considered better than the current system of only receiving antivirals after an infection.

Further to this, the project will look to investigate partial pre-distribution and attempt to derive optimality in terms of the proportion of households to pre-distribute antivirals to.